\documentclass[ucs,9pt]{beamer}

% Set input encoding to UTF-8.
\usepackage[utf8x]{inputenc}

% Set fonts.
\usepackage[T1]{fontenc}
\usepackage{textcomp}
\usepackage{tgtermes}
\usepackage{tgheros}
\usepackage{tgcursor}

% Set language.
\usepackage[english]{babel}

\usepackage{microtype}
\usepackage{listings}

\title{EScript (0.7.2 Egon)}
\subtitle{A Short Introduction Part II -- The EScript StdLib}
\author{Claudius~Jähn}
\date{Version: March, 2015}
\titlegraphic{\includegraphics[width=0.1\textwidth]{logoAscii.pdf}}

% theme Benjamin
%\useinnertheme{rectangles}
%\useoutertheme{infolines}
%\usecolortheme{seahorse}
%\usecolortheme{rose}

% theme Claudius
\usetheme{Pittsburgh}
\useinnertheme{rectangles}
\usecolortheme{fly}
\setbeamercolor{normal text}{bg=black, fg=white}
\setbeamercolor{subtitle}{bg=black, fg=white}
\setbeamercolor{block body}{fg=white,bg=white!10!black}

\AtBeginSection[] {
\begin{frame}{Overview}
	\tableofcontents[currentsection]
\end{frame}
}

\begin{document}
\lstdefinelanguage{EScript}{
	morekeywords={var,new,fn,this,true,false,void,if,else,while,for,foreach,as,thisFn,break,continue,return,once,init,switch,case,yield,static},
	morekeywords=[2]{out,outln,ExtObject,Type,Number,Map,Array},
	sensitive=true,
	morecomment=[l]{//},
	morecomment=[s]{/*}{*/},
	morestring=[b]{"},
	morestring=[d]{’}
}
\lstset{
	language=EScript,
	showstringspaces=false,
	tabsize=4,
	basicstyle=\ttfamily,
	%keywordstyle=[2]\color{blue}
}
\maketitle


\begin{frame}{Overview}
\tableofcontents
\end{frame}

%%%%%%%%%%%%%%%%%%%%%%%%%%%%%%%%%%%%%%%%%%%%%%%%%%%%%
\section{Introduction}
\begin{frame}{Part II - The EScript StdLib}
\begin{itemize}
		\item Set of helper functions and types.
		\item Goals: 
		\begin{itemize}
			\item Provide helper functions and additional data types.
			\item Provide method to customize objects and types using \textbf{traits}.
			\item Support project structuring using \textbf{modules}.
		\end{itemize}
\end{itemize}
\end{frame}



\begin{frame}[fragile]{Init the StdLib}
\begin{itemize}
		\item (quick and easy) Load the complete StdLib into the global Std-namespace:
\begin{lstlisting}
load("./Std/complete.escript");

// Std - contains all components of the StdLib.
// example:
var myMultiprocedure = new Std.MultiProcedure;
\end{lstlisting}
		\item Alternativ: Load the module loader and load modules when needed.
\begin{lstlisting}
var module = load("./Std/module.escript");
// example:
var MultiProcedure = module('Std/MultiProcedure');
var myMultiprocedure = new MultiProcedure;
\end{lstlisting}
\end{itemize}
\end{frame}

\section{Modules}

\begin{frame}[fragile]{Modules: Overview}
\begin{itemize}
\item Modules are arbitrary objects identified by an unique id (e.g. 'Std/MultiProcedure').
\item A Module is (normally) defined in a single file.
\item The module loader loads the file when first needed.
\item The concept of EScript modules is based on the JavaScript AMD (Asynchronous Module Definition).
\item Goals of using modules:
\begin{itemize}
\item Helps structuring code
\item Only load necessary files when needed
\item Explicitly defines dependencies between modules 
\item No namespace pollution 
\end{itemize}
\item Note for experienced EScript users: Modules are the recommended Replacement for  \lstinline!loadOnce!.
\end{itemize}
\end{frame}


\begin{frame}[fragile]{Modules: The module loader}
How to get a module loader object:
\begin{itemize}
\item \lstinline!var module = load("PathToStdLib/module.escript"); !
\item If the StdLib has been registered to the Std-Namespace, a module loader is avaiable as \lstinline!Std.module! (going to change to \lstinline!Std.module!)
\item If the file is loaded by a module loader itself, a module loader is available as injected static variable: \lstinline!module!\\
The injected module loader supports relative module ids.
\end{itemize}
Usage:
\begin{itemize}
\item \lstinline!var MultiProcedure = module('Std/MultiProcedure');!
\end{itemize}
\end{frame}

\begin{frame}[fragile]{Modules: Writing a module}
Modules are defined by ordinary EScript files returning the module value.
\begin{lstlisting}
// MyTypeAsModule.escript
var T = new Type;
T.makeFoo ::= fn(){	outln("foo"); };
return T;

// main.escript
// (module loader is available in Std namespace)
var MyType = Std.module('MyTypeAsModule);
var myObject = new MyType;
myObject.makeFoo();
\end{lstlisting}
 
\end{frame}


\begin{frame}[fragile]{Modules: Module callbacks}
\begin{itemize}
\item Loading a module can trigger a callback.\\
 \lstinline!module.on('MyModule',fn(MyModule){doSomething(MyModule);})
\item If the module has already been loaded, the callback is executed immediately.
\item Application scenario: Optional extension
\begin{lstlisting}
// myFeature.escript -- a module that 
//  registers its own gui entry iff the 
//  MySystem/gui is available

static myFeature;
...
module.on( 'MySystem/gui', fn(gui){
	gui.addSomeFancyMenuEntries( myFeature );
});
return myFeature;

\end{lstlisting}
\end{itemize}
\end{frame}

% ----------------------------------------------------------------

\section{Traits}

\begin{frame}[fragile]{Traits: Motivation}
\begin{itemize}
\item Challenge: Create objects/types that provide a mix of different properties.
\item Possible solutions: Multiple inheritance; interfaces; composition.
\item Possible solution in EScript: Add desired properties/members/methods manually as object attributes.
\item Solution provided by StdLib: \textbf{Traits} 
\item A Trait adds a \textbf{well defined} property/functionality to an object or a type.
\end{itemize}
\end{frame}

\begin{frame}[fragile]{Traits: Example 1/1}
Example: 
\begin{lstlisting}
// Property 1: 
//	has function "makeFoo()" 
//		as defined in FooMakerTrait.
// Property 2: 
//	has function "makeBar()" 
//		as defined in BarMakerTrait.

var myObject = new ExtObject;
Std.Traits.addTrait( myObject, FooMakerTrait );
Std.Traits.addTrait( myObject, BarMakerTrait, 2 );
myObject.makeFoo(); // output: foo
myObject.makeBar(); // output: barbar
\end{lstlisting}
\end{frame}

\begin{frame}[fragile]{Traits: Example 2/2}
\begin{lstlisting}
var FooMakerTrait = new Std.Traits.GenericTrait;
FooMakerTrait.attributes.makeFoo := fn(){ 
	outln("foo"); 
};

// ------------

var BarMakerTrait = new Std.Traits.GenericTrait;
BarMakerTrait.attributes.makeBar := fn(){ 
	outln("bar"*this._numBars); 
};
BarMakerTrait.onInit += fn(obj, Number numBars){
	obj._numBars@(private) := numBars;
};
\end{lstlisting}
\end{frame}


\begin{frame}[fragile]{Traits: Type Example}
\begin{lstlisting}
var BlubMakerTrait = new Std.Traits.GenericTrait;
BlubMakerTrait.attributes.makeBlub ::= fn(){ 
	outln("blub"*this.numBlubs); 
};
BlubMakerTrait.attributes.numBlubs := 3;
BlubMakerTrait.attributes.someArray @(init) := 
								Array;
// Restrict trait to Types:
BlubMakerTrait.onInit += fn(Type t){
}; 


var MyType = new Type;
Std.Traits.add( MyType, BlubMakerTrait);
var a = new MyType;
a.makeBlub();
...

\end{lstlisting}
\end{frame}
%

\begin{frame}[fragile]{Traits: Functions}
\begin{itemize}
\item \lstinline!Std.Traits.addTrait(obj, trait, parameters...)! \\
	Add the given trait to the given object; additional parameters are passed to the trait's \lstinline!onInit! method.
\item \lstinline!Std.Traits.queryTrait(obj, trait)! \\
	Checks if \lstinline!obj! has the given trait.
\item \lstinline!Std.Traits.requireTrait(obj, trait)! \\
	Throw an error if \lstinline!obj! does not have the given trait.
\item \lstinline!Std.Traits.assureTrait(obj, trait)! \\
	If \lstinline!obj! does not have the given trait, add it now.
\end{itemize}
\end{frame}
%


%
\begin{frame}[fragile]{Other modules...}
\begin{itemize}
\item MultiProcedure
\item DataWrapper
\item JSon-DataStore
\item Set
\item PriorityQueue
\item ObjectSerialization
\item addRevocably
\end{itemize}
\end{frame}

\begin{frame}[fragile]{Part III -- EScript and PADrend}
\begin{itemize}
\item Was ist ein Plugin (im Vergleich zu einem Modul)? 
\item Was ist ein Extension-Point?
\item GUI (Popup-Window, DataWrapper, Menu, Registry, Button)
\item Beispiel-Plugin
\item Plugin-Struktur von PADrend
\item Wie läuft PADrend? Starten/Initialisieren/EventHandling/Rendering
\end{itemize}
\end{frame}
%
%
%%%%%%%%%%%%%%%%%%%%%%%%%%%%%%%%%%%%%%%%%%
%\section{Std library}
%\begin{frame}[fragile]{Std library}
	%\begin{itemize}
		%\item Example: \lstinline!MultiProcedure!
		%\begin{lstlisting}
%var myFun = new Std.MultiProcedure;
%
%myFun += fn(arr){	arr += 'Hallo'; };
%myFun += fn(arr){	
   %arr += ' Welt'; 
   %return $REMOVE;
%};  
%myFun += fn(arr){	arr += '';  };
%
%var arr = [];
%myFun( arr );
%outln(arr.implode() ); // Outputs: Hallo Welt!
%
%var arr2 = [];
%myFun( arr2 );
%outln(arr2.implode() ); // Outputs: Hallo!
%
		%\end{lstlisting}
	%\end{itemize}
	%
%\end{frame}
%%
%%%%%%%%%%%%%%%%%%%%%%%%%%%%%%%%%%%%%%%%%%%%%%%


\end{document}