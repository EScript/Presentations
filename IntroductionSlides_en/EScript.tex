\documentclass[ucs,9pt]{beamer}

% Set input encoding to UTF-8.
\usepackage[utf8x]{inputenc}

% Set fonts.
\usepackage[T1]{fontenc}
\usepackage{textcomp}
\usepackage{tgtermes}
\usepackage{tgheros}
\usepackage{tgcursor}

% Set language.
\usepackage[english]{babel}

\usepackage{microtype}
\usepackage{listings}

\title{EScript}
\subtitle{A Short Introduction}
\author{Benjamin~Eikel, Claudius~Jähn}
\date{Version: June 16, 2012}
\titlegraphic{\includegraphics[width=0.1\textwidth]{logoAscii.pdf}}

% theme Benjamin
%\useinnertheme{rectangles}
%\useoutertheme{infolines}
%\usecolortheme{seahorse}
%\usecolortheme{rose}

% theme Claudius
\usetheme{Pittsburgh}
\useinnertheme{rectangles}
\usecolortheme{fly}
\setbeamercolor{normal text}{bg=black, fg=white}
\setbeamercolor{subtitle}{bg=black, fg=white}
\setbeamercolor{block body}{fg=white,bg=white!10!black}

\AtBeginSection[] {
\begin{frame}{Overview}
	\tableofcontents[currentsection]
\end{frame}
}

\begin{document}
\lstdefinelanguage{EScript}{
	morekeywords={var,new,fn,this,true,false,void,if,else,while,for,foreach,as,thisFn,break,continue,return,once,init,switch,case,yield},
	morekeywords=[2]{out,outln,ExtObject,Type,Number,Map,Array},
	sensitive=true,
	morecomment=[l]{//},
	morecomment=[s]{/*}{*/},
	morestring=[b]{"},
	morestring=[d]{’}
}
\lstset{
	language=EScript,
	showstringspaces=false,
	tabsize=4,
	basicstyle=\ttfamily,
	%keywordstyle=[2]\color{blue}
}
\maketitle


\begin{frame}{Overview}
\tableofcontents
\end{frame}

%%%%%%%%%%%%%%%%%%%%%%%%%%%%%%%%%%%%%%%%%%%%%%%%%%%%%
\section{Introduction}
\begin{frame}[t]{What is EScript?}
\includegraphics[width=1cm]{EScript_Logo} EScript $\ldots$ \\[1em]
\only<1>{
\begin{itemize}
	\addtolength{\itemsep}{\baselineskip}
	\item is an object-oriented scripting language.
	\item is compiled and executed by a virtual machine.
	\item has a similar syntax to C.
	\item was developed to use C++ objects from scripts easily.
\end{itemize}
}
\only<2>{
\begin{itemize}
	\addtolength{\itemsep}{\baselineskip}
	\item is released under a free software license (MIT).
	\item is available from \url{https://github.com/EScript}.
	\item has a command-line interpreter.
	\item can be built using CMake.
	\item can be used internally by other C++ projects \\
	(e.g. by PADrend \url{http://PADrend.de}).
	\item stands for HasE-Script.
\end{itemize}
}
\end{frame}

\begin{frame}[fragile]{First Example}
\begin{itemize}
	\addtolength{\itemsep}{\baselineskip}
	%\item EScript files should have the extension \texttt{.escript}.
	%\item The EScript parser analyzes the script file line by line.
	\item A simple script: \lstinline{HelloWorld.escript}
	\small
			\begin{lstlisting}
				outln( "Hello, world!" ); // Outputs: Hello World!
			\end{lstlisting}
	\normalsize
		\item Calls the global function \lstinline!out! with the string  \lstinline{"Hello world!"}  as parameter value.
		\item The statements ends with a semicolon.
		\item Comments begin with  \lstinline{//}  or are enclosed \lstinline{/*    */}.
\end{itemize}
\end{frame}

%%%%%%%%%%%%%%%%%%%%%%%%%%%%%%%%%%%%%%%%%%%%%%%%%%%%%

\section{Data Types and Operators}
\begin{frame}[fragile]{Simple Types (call-by-value)}
	\begin{block}{Number}
		\hfill \lstinline!42! \hfill \lstinline!27.4! \hfill \lstinline!0x1a! \hfill \lstinline!-2.7e+10! \hfill \lstinline!0b101010! \hfill{}
	\end{block}
	\begin{block}{String}
		\hfill \lstinline!"an"! \hfill \lstinline!'example string'!  \hfill{}
	\end{block}
	\begin{block}{Bool}
		\hfill \lstinline!true! \hfill \lstinline!false! \hfill{}
	\end{block}
	\begin{block}{Void (empty or no value)}
		\hfill \lstinline!void! \hfill{}
	\end{block}
\end{frame}

\begin{frame}[fragile]{Operators}
	\begin{block}{Some operators}
				\begin{lstlisting}
				outln( 2+40 ); // Output: 42
				outln( 2*21 ); // Output: 42
				outln( "4" + "2" ); // Output: 42
				outln( "foo"+"'bar" ); // Output: foobar
				outln( "wup " * (6/2)  ); // Output: wup wup wup
				outln( 1>2  ); // Output: false
				outln( !true   ); // Output: false
				outln( true & true  ); // Output: true
				outln( false || true  ); // Output: true
				outln( "foo" == "bar"'  ); // Output: false
				outln( "foo" != "bar"'  ); // Output: true
			\end{lstlisting}
	\end{block}
\end{frame}
	
\begin{frame}[fragile]{Type Conversion}
	\begin{block}{Only \lstinline!false!  and \lstinline!void!  convert to \lstinline!false! }
		\begin{lstlisting}
			outln(false || false); // Output: false
			outln(false || void); // Output: true
			outln(false || 0); // Output: true
			outln(false || ""); // Output: true
		\end{lstlisting}
	\end{block}
	\pause
	\begin{block}{Conversion to number (left operand is a number)}
		\begin{lstlisting}
			outln( 12 + "3" ); // Output: 15
			outln( 10 * "10" ); // Output: 100
			outln( 10 == "10" ); // Output: true
			outln( 10 == "10.0" ); // Output: true
		\end{lstlisting}
	\end{block}
	\pause
	\begin{block}{Conversion to string (left operand is a string)}
		\begin{lstlisting}
			outln("12" + 3); // Output: 123
			outln("10" == 10); // Output: true
			outln("10.0" == 10); // Output: false
		\end{lstlisting}
	\end{block}
\end{frame}

\begin{frame}[fragile]{Equality checks}
		\begin{block}{Check equality with conversion == \\
		Check equality without conversion === }
		\begin{lstlisting}
			outln( 10 == "10" ); // Output: true
			outln( 10 === "10" ); // Output: false
			outln( 10 === 10 ); // Output: true
			outln( true == "foo" ); // Output: true
			outln( true === "true" ); // Output: false
			outln( "true" == true ); // Output: true
			outln( "true" === true ); // Output: false
		\end{lstlisting}
	\end{block}
\end{frame}
%			outln((60 + "4").sqrt()); // Output: 8
			%outln((10 * "10").log(10)); // Output: 2
%%%%%%%%%%%%%%%%%%%%%%%%%%%%%%%%%%%%%%%%%%%%%%%%%%%%%

\section{Functions and variables}
\begin{frame}[fragile]{Calling functions}
	\begin{block}{Calling functions with different origins}
		\begin{lstlisting}
			// call global function 'load':
			load( "someScript.escript" );  
			
			// call function 'saveTextFile' in namespace 'IO':
			IO.saveTextFile( "foo.txt" , "bar" ); 
			
			// call member function 'sqrt' of object 9.0:
			out( (9.0).sqrt() ); // Output: 3
		\end{lstlisting}
	\end{block}
\end{frame}

%%%%%%%%%%%%%%%%%%%%%%%%%%%%%%%%%%%%%%%%%%%%%%%%%%%%%
	
\begin{frame}[fragile]{Declaring Variables}
	\begin{block}{Declare a variable with \lstinline!var!:}
		\begin{lstlisting}
			
			// "foo" is an empty variable (contains void).
			var foo; 
			
			// The variable "xPos" contains a number
			var xPos = 500 - 80 / 2;

			// The variable "message" will be of type String
			var message = "Please click the button";

			// Dynamically change the type to Number
			message = 5;
		\end{lstlisting}
	\end{block}
\end{frame}

\begin{frame}[fragile]{Declaring simple functions}
	\begin{itemize}
	\item Declare functions with \lstinline!fn!.
	\item Functions have no names, but they can be stored in a variable:
		\begin{lstlisting}
			var square = fn(num) {
			    return num * num;
			};
			var a = square(5);
			var b = square(4.2);
		\end{lstlisting}
		\end{itemize}
	\end{frame}

\begin{frame}[fragile]{Advanced Types (1)}
	\begin{block}{Array}
		\begin{lstlisting}
			var numbers = [3, 23, 7, 3, 100, 1, 35];
			var colors = ["red", "green", "blue"];
			outln( numbers[4] ); // Outputs: 100
			outln( numbers.count() ); // Outputs: 7
			outln( numbers.empty() ); // Outputs: false
		\end{lstlisting}
	\end{block}
	\begin{block}{Map}
		\begin{lstlisting}
			var fruits = {
			    "lemon" : "yellow",
			    "cherry" : "red"
			};
			fruits["apple"] = "green";
		\end{lstlisting}
	\end{block}
\end{frame}

%%%%%%%%%%%%%%%%%%%%%%%%%%%%%%%%%%%%
\section{Control Structures}
\begin{frame}[fragile]{Conditionals (1)}
	\begin{block}{Conditional execution with \lstinline!if/else!.}
		\begin{lstlisting}
			var result = /* some function */;
			if(result) {
			    out("Success");
			} else {
			    out("Failure");
			}
		\end{lstlisting}
		\begin{lstlisting}
			var num = /* some number */;
			if(num < 0) {
			    out("Too small");
			} else if(num >= 0 && num <= 100) {
			    out("Range okay");
			} else {
			    out("Too large");
			}
		\end{lstlisting}
	\end{block}
\end{frame}

\begin{frame}[fragile]{Conditionals (2)}
	\begin{block}{? (conditional operator)}
		\begin{lstlisting}
			var num = /* some number */;
			var positive = (num > 0) ? true : false;
		\end{lstlisting}
	\end{block}
	\pause
	\vfill
\end{frame}

\begin{frame}[fragile]{Loops (1)}
	\begin{block}{Looping with \lstinline!while!:}
		\begin{lstlisting}
			var tasks = [/* some tasks */];
			while(!tasks.empty()) {
			    var firstTask = tasks.front();
			    tasks.popFront();
			    // do something with first task
			}
		\end{lstlisting}
	\end{block}
\end{frame}

\begin{frame}[fragile]{Loops (2)}
	\begin{block}{Looping with \lstinline!for!:}
		\begin{lstlisting}
			var sum = 0;
			for(var i = 0; i < 100; ++i) {
			    sum += i;
			}
			out("Sum of numbers: ", sum, "\n");
		\end{lstlisting}
	\end{block}
\end{frame}

\begin{frame}[fragile]{Loops (3)}
	\begin{block}{Iterate over a container: \lstinline!foreach!.}
		\begin{lstlisting}
			var chars = ["a", "c", "k", "b", "d", "x", "j"];
			foreach(chars as var i, var c) {
			    if(c === "x") {
			        out("Character 'x' found at index " + i);
			        break;
			    }
			}
		\end{lstlisting}
	\end{block}
	Output: Character "x" found at index 5
\end{frame}

%%%%%%%%%%%%%%%%%%%%%%%%%%%%%%%%%%%%%%%%
\section{Objects and Types}
\begin{frame}[fragile]{Extendable object}
	\begin{block}{Extendable objects:  \lstinline!ExtObject!.}
		\begin{lstlisting}
			var car = new ExtObject;
			car.color := "red"; // create new member
			car.speed := 190;
			car.outputDesc := fn() {
			    out("This is a ", this.color, " car ");
			    out("with top speed ", this.speed, ".\n");
			};

			...

			car.speed = 185;
			car.outputDesc();
		\end{lstlisting}
	\end{block}
	Output: This is a red car with top speed 185.
\end{frame}

\begin{frame}[fragile]{Inheritance}
	\begin{block}{Types and inheritance:}
		\begin{lstlisting}
			var Shape = new Type;
			Shape.color := "white";

			// New type that is derived from Shape
			var Polygon = new Type(Shape); 
			Polygon.numVertices := 3;

			// New type that is derived from Shape
			var Circle = new Type(Shape); 
			Circle.radius := 0;

			var circle = new Circle;
			circle.color = "red";
			circle.radius = 5;
		\end{lstlisting}
	\end{block}
\end{frame}
 
%%%%%%%%%%%%%%%%%%%%%%%%%%%%%%%%%%%%%%%%%%%%%%

\section{Other Features}
\begin{frame}{Delegation}
	Call a function on another object.
	\begin{block}{Example}
		\lstinputlisting{Delegation.escript}
	\end{block}
\end{frame}

\begin{frame}{Properties}
	\begin{block}{Example}
		\lstinputlisting{Properties.escript}
	\end{block}
\end{frame}

\section{Examples}
\begin{frame}{Factorial}
	Factorial: $\qquad n! = 1 \cdot 2 \cdot 3 \cdot \ldots \cdot n \qquad 0! = 1$
	\begin{block}{Example}
		\lstinputlisting{ExampleFactorial.escript}
	\end{block}
\end{frame}

\begin{frame}{Player}
	\begin{block}{Example}
		\lstinputlisting{ExamplePlayer.escript}
	\end{block}
\end{frame}

\begin{frame}{Further Documentation}
	You can find additional documentation in \texttt{EScript/docs/Introduction.html}.
\end{frame}


% binding parameters/Objects
% var and static life
% attributes
% exception handling
% std lib
% modules
% traits

%homework

\end{document}