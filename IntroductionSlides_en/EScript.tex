\documentclass[ucs,9pt]{beamer}

% Set input encoding to UTF-8.
\usepackage[utf8x]{inputenc}

% Set fonts.
\usepackage[T1]{fontenc}
\usepackage{textcomp}
\usepackage{tgtermes}
\usepackage{tgheros}
\usepackage{tgcursor}

% Set language.
\usepackage[english]{babel}

\usepackage{microtype}
\usepackage{listings}

\title{EScript (0.7.2 Egon)}
\subtitle{A Short Introduction}
\author{Benjamin~Eikel, Claudius~Jähn}
\date{Version: March, 2015}
\titlegraphic{\includegraphics[width=0.1\textwidth]{logoAscii.pdf}}

% theme Benjamin
%\useinnertheme{rectangles}
%\useoutertheme{infolines}
%\usecolortheme{seahorse}
%\usecolortheme{rose}

% theme Claudius
\usetheme{Pittsburgh}
\useinnertheme{rectangles}
\usecolortheme{fly}
\setbeamercolor{normal text}{bg=black, fg=white}
\setbeamercolor{subtitle}{bg=black, fg=white}
\setbeamercolor{block body}{fg=white,bg=white!10!black}

\AtBeginSection[] {
\begin{frame}{Overview}
	\tableofcontents[currentsection]
\end{frame}
}

\begin{document}
\lstdefinelanguage{EScript}{
	morekeywords={var,new,fn,this,true,false,void,if,else,while,for,foreach,as,thisFn,break,continue,return,once,init,switch,case,yield,static},
	morekeywords=[2]{out,outln,ExtObject,Type,Number,Map,Array},
	sensitive=true,
	morecomment=[l]{//},
	morecomment=[s]{/*}{*/},
	morestring=[b]{"},
	morestring=[d]{’}
}
\lstset{
	language=EScript,
	showstringspaces=false,
	tabsize=4,
	basicstyle=\ttfamily,
	%keywordstyle=[2]\color{blue}
}
\maketitle


\begin{frame}{Overview}
\tableofcontents
\end{frame}

%%%%%%%%%%%%%%%%%%%%%%%%%%%%%%%%%%%%%%%%%%%%%%%%%%%%%
\section{Introduction}
\begin{frame}[t]{What is EScript?}
\includegraphics[width=1cm]{EScript_Logo} EScript $\ldots$ \\[1em]
\only<1>{
\begin{itemize}
	\addtolength{\itemsep}{\baselineskip}
	\item is an object-oriented scripting language.
	\item is compiled and executed by a virtual machine.
	\item has a similar syntax to C.
	\item was developed to use C++ objects from scripts easily.
\end{itemize}
}
\only<2>{
\begin{itemize}
	\addtolength{\itemsep}{\baselineskip}
	\item is released under a free software license (MIT).
	\item is available from \url{https://github.com/EScript}.
	\item has a command-line interpreter.
	\item can be built using CMake.
	\item can be used internally by other C++ projects \\
	(e.g. by PADrend \url{http://PADrend.de}).
	\item stands for HasE-Script.
\end{itemize}
}
\end{frame}

\begin{frame}[fragile]{First Example}
\begin{itemize}
	\addtolength{\itemsep}{\baselineskip}
	\item A simple script:
	%\\
	\small
			\begin{lstlisting}
outln( "Hello World!" ); // Outputs: Hello World!
			\end{lstlisting}
	\normalsize
	\pause
		\item Calls the global function \lstinline!outln! with the string  \lstinline{"Hello World!"}  as parameter value.
		\item The statements ends with a semicolon.
		\item Comments begin with  \lstinline{//}  or are enclosed with \lstinline{/*    */}.
\end{itemize}
\end{frame}

%%%%%%%%%%%%%%%%%%%%%%%%%%%%%%%%%%%%%%%%%%%%%%%%%%%%%

\section{Data Types and Operators}
\begin{frame}[fragile]{Simple Types (call-by-value)}
	\begin{block}{Number}
		\hfill \lstinline!42! \hfill \lstinline!27.4! \hfill \lstinline!0x1a! \hfill \lstinline!-2.7e+10! \hfill \lstinline!0b101010! \hfill{}
	\end{block}
	\begin{block}{String}
		\hfill \lstinline!"an"! \hfill \lstinline!'example string'!  \hfill{}
	\end{block}
	\begin{block}{Bool}
		\hfill \lstinline!true! \hfill \lstinline!false! \hfill{}
	\end{block}
	\begin{block}{Void (empty or no value)}
		\hfill \lstinline!void! \hfill{}
	\end{block}
\end{frame}

\begin{frame}[fragile]{Operators}
	\begin{block}{Some operators}
\begin{lstlisting}
outln( 2+40 ); // Output: 42
outln( 2*21 ); // Output: 42
outln( "4" + "2" ); // Output: 42
outln( "foo"+"bar" ); // Output: foobar
outln( "wup " * (6/2)  ); // Output: wup wup wup
outln( 1>2  ); // Output: false
outln( !true   ); // Output: false
outln( true & true  ); // Output: true
outln( false || true  ); // Output: true
outln( "foo" == "bar"  ); // Output: false
outln( "foo" != "bar"  ); // Output: true
\end{lstlisting}
	\end{block}
\end{frame}
	
\begin{frame}[fragile]{Type Conversion}
	\begin{block}{Only \lstinline!false!  and \lstinline!void!  convert to \lstinline!false! }
		\begin{lstlisting}
outln(false || false); // Output: false
outln(false || void); // Output: false
outln(false || 0); // Output: true
outln(false || ""); // Output: true
		\end{lstlisting}
	\end{block}
	\pause
	\begin{block}{Conversion to number (left operand is a number)}
		\begin{lstlisting}
outln( 12 + "3" ); // Output: 15
outln( 10 * "10" ); // Output: 100
outln( 10 == "10" ); // Output: true
outln( 10 == "10.0" ); // Output: true
		\end{lstlisting}
	\end{block}
	\pause
	\begin{block}{Conversion to string (left operand is a string)}
		\begin{lstlisting}
outln( "12" + 3 ); // Output: 123
outln( "10" == 10 ); // Output: true
outln( "10.0" == 10 ); // Output: false
		\end{lstlisting}
	\end{block}
\end{frame}

\begin{frame}[fragile]{Equality checks}
		\begin{block}{Check equality with conversion == \\
		Check equality without conversion === }
		\begin{lstlisting}
outln( 10 == "10" ); // Output: true
outln( 10 === "10" ); // Output: false
outln( 10 === 10 ); // Output: true
outln( true == "foo" ); // Output: true
outln( true === "true" ); // Output: false
outln( "true" == true ); // Output: true
outln( "true" === true ); // Output: false
		\end{lstlisting}
	\end{block}
\end{frame}


\begin{frame}[fragile]{Special type: Identifier}
\begin{itemize}
	\item Variable and attribute names have a special data type: Identifier.
	\item Identifier objects are immutable (can not be changed).
	\item Identifiers are often used as value for constants.
	\item Identifiers are created using the dollar sign: \lstinline!$exampleIdentifier!
	\item \lstinline!$foo == "foo" // false!
	\item \lstinline!"foo" == $foo // true!
\end{itemize}
\end{frame}

%%%%%%%%%%%%%%%%%%%%%%%%%%%%%%%%%%%%%%%%%%%%%%%%%%%%%%

\section{Calling functions}
\begin{frame}[fragile]{Calling functions}
	\begin{block}{Calling functions with different origins:}
		\begin{lstlisting}
// call global function 'load':
load( "someScript.escript" );  

// call function 'saveTextFile' in namespace 'IO':
IO.saveTextFile( "foo.txt" , "bar" ); 

// call method 'sqrt' of object 9.0:
out( (9.0).sqrt() ); // Output: 3
		\end{lstlisting}
	\end{block}
\end{frame}
%
%%%%%%%%%%%%%%%%%%%%%%%%%%%%%%%%%%%%%%%%%%%%%%%%%%%%%%

\section{Local variables}
\begin{frame}[fragile]{Declaring Variables}
	\begin{block}{Declare a variable with \lstinline!var!:}
		\begin{lstlisting}
			
// "foo" is an empty variable (containing void).
var foo; 

// The variable "xPos" contains a number
var xPos = 500 - 80 / 2;

// The variable "message" will be of type String
var message = "Please click the button";

// Dynamically change the type to Number
message = 5;
		\end{lstlisting}
	\end{block}
\end{frame}

\section{Arrays and Maps}
\begin{frame}[fragile]{Built-in collection types}
	\begin{block}{Array}
		\begin{lstlisting}
var numbers = [3, 23, 7, 3, 100, 1, 35];
var colors = ["red", "green", "blue"];
outln( numbers[4] ); // Outputs: 100
outln( numbers.count() ); // Outputs: 7
outln( numbers.empty() ); // Outputs: false
		\end{lstlisting}
	\end{block}
	\begin{block}{Map}
		\begin{lstlisting}
var fruits = {
		"lemon" : "yellow",
		"cherry" : "red"
};
fruits["apple"] = "green";
		\end{lstlisting}
	\end{block}
\end{frame}

%%%%%%%%%%%%%%%%%%%%%%%%%%%%%%%%%%%%%
\section{Control Structures}
\begin{frame}[fragile]{Conditionals (1)}
	\begin{block}{Conditional execution with \lstinline!if/else!.}
		\begin{lstlisting}
var result = someFunction();
if(result) {
		out("Success");
} else {
		out("Failure");
}
		\end{lstlisting}
		\begin{lstlisting}
var num = calculateSomething();
if(num < 0) 
		out("Too small");
else if(num >= 0 && num <= 100)
		out("Range okay");
else
		out("Too large");
		\end{lstlisting}
	\end{block}
\end{frame}

\begin{frame}[fragile]{Conditionals (2)}
	\begin{block}{? (conditional operator)}
		\begin{lstlisting}
var num = calculateSomething();
var positive = (num > 0) ? true : false;
		\end{lstlisting}
	\end{block}
\end{frame}

\begin{frame}[fragile]{Loops (1)}
	\begin{block}{Looping with \lstinline!while!:}
		\begin{lstlisting}
var numbers = [ 4,5,29,32 ];
while(!numbers.empty()) {
		var n = tasks.back();
		n.popBack();
		out( n, " " );
}
// Outputs: 32 29 5 4
		\end{lstlisting}
	\end{block}
\end{frame}

\begin{frame}[fragile]{Loops (2)}
	\begin{block}{Looping with \lstinline!for!:}
		\begin{lstlisting}
var sum = 0;
for(var i = 0; i < 100; ++i) {
		sum += i;
}
outln("Sum of numbers: ", sum);
		\end{lstlisting}
	\end{block}
\end{frame}

\begin{frame}[fragile]{Loops (3)}
	\begin{block}{Iterate over a container: \lstinline!foreach!.}
		\small	
		\begin{lstlisting}
var chars = ['a', 'c', 'k', 'b', 'd', 'x', 'j'];
foreach(chars as var index, var currentChar) {
		if(currentChar === 'x') {
				outln("Character 'x' found at index ", 
					index);
				break;
		}
}
		\end{lstlisting}
		\normalsize
	\end{block}
	\begin{itemize}	
	\item Output: Character 'x' found at index 5
	\item The index variable is optional:
	 \lstinline!foreach( collection as var value) outln(value);!
	\end{itemize}
\end{frame}


\begin{frame}[fragile]{Exception handling}
	\begin{block}{Catch and handle an exception: \lstinline!try/catch!.}
		\small	
		\begin{lstlisting}
try{
    outln(  42/0 );
}catch(e){
    outln( e );
}
		\end{lstlisting}
		\normalsize
	\end{block}
	\begin{itemize}	
	\item Output: Division by zero...
	\item For throwing an exception, use \lstinline!Runtime.exception('message');!
	\end{itemize}
\end{frame}

%%%%%%%%%%%%%%%%%%%%%%%%%%%%%%%%%%%%%%%%%
\section{Functions}
\begin{frame}[fragile]{Declaring simple functions}
	\begin{itemize}
	\item Declare functions with \lstinline!fn!
	\item Functions have no names, but they can be stored in a variable:
		\begin{lstlisting}
var square = fn(num) {
		return num * num;
};
var a = square(5);
outln( a ); // Outputs: 25
		\end{lstlisting}
		\end{itemize}
\end{frame}


\begin{frame}[fragile]{Parameters}
	\begin{itemize}
	\item Parameters can be restricted with type checks:
		\begin{lstlisting}
var square = fn(Number num) {
		return num * num;
};
square(4); // ok
square( 'foo' ); // runtime error
		\end{lstlisting}
	\pause
	\item Parameters can have default values:
		\begin{lstlisting}
var add = fn(a,b=1) {
		return a+b;
};
outln( add( 10,2 ) ); // Outputs: 12
outln( add( 10 ) ); // Outputs: 11
		\end{lstlisting}
	\end{itemize}
\end{frame}

\begin{frame}[fragile]{Multi parameters}
	\begin{itemize}
	\item Multi parameters accept arbitrary many values and store them in an array:
		\begin{lstlisting}
var sum = fn( numbers... ) {
	var sum = 0;
	foreach( numbers as var n)
		sum += n;
	return sum;
};
outln( sum( 10,100,1000,4 ) ); // Outputs 1114
		\end{lstlisting}
		\end{itemize}
\end{frame}

\begin{frame}[fragile]{Multi assignment}
	\begin{itemize}
	\item To handle multiple return values, arrays can be assigned to multiple variables at once: 
	 \lstinline![ lValue* ] = Array!.
		\begin{lstlisting}
var calc = fn() {
	var result1 = 17;
	var result2 = 42;
	return [17,42];
};
[var a, var b] = calc();
outln( "a:",a,"b:",b );
		\end{lstlisting}
		\end{itemize}
\end{frame}


\begin{frame}[fragile]{Bind parameter values}
	\begin{itemize}
	\item Set the first parameters to fixed values: \lstinline!Array => fn(...)!.
	\item Bound function object behaves like normal function.
%	\item Functions have no names, but they can be stored in a variable:
		\begin{lstlisting}
var myFun = fn(a,b,c) {
    out( 'a:', a, ' b:', b, ' c:' c );
};
myFun( 1, 2, 3);  // Output: a:1 b:2 c:3

var myBoundFun = [ 100, 200 ] => myFun;
myBoundFun( 300 ); // Output: a:100 b:200 c:300
		\end{lstlisting}
		\end{itemize}
\end{frame}

\begin{frame}[fragile]{Bind calling object}
	\begin{itemize}
	\item Create a combination of a function and and object: \lstinline!object->fun!
	\item Bound function object behaves like normal function.
	\item When called, the bound object is the function's \lstinline!this!-object.
%	\item Functions have no names, but they can be stored in a variable:
		\begin{lstlisting}
var myObject = [100,200];
var myBoundFun = myObject -> Array.max;

outln( myBoundFun() ); // Output: 200
		\end{lstlisting}
		\end{itemize}
\end{frame}

\begin{frame}[fragile]{Variable and parameter scopes}
	\begin{itemize}
	\item The scope of local variables (\lstinline!var!) is the tightest enclosing block, but \emph{excluding} functions defined in the block.
	\item The scope of a parameter is the enclosing block of the function, but \emph{excluding} functions defined in the function.
	\item Local variables and parameters are allocated for every call of the containing function (on a stack).
		\pause
	\item The scope of static variables (\lstinline!static!) is the tightest enclosing block, \emph{including} functions defined in the block.
	\item Static variables are allocated once for all calls of the containing function.
	\pause
			\begin{lstlisting}
static factorial = fn( Number n ) {
    return (n == 0) ? 1 : factorial (n - 1) * n;
};
out( factorial( 5 ) ); // Output: 120
		\end{lstlisting}
	\end{itemize}
		
\end{frame}

%%%%%%%%%%%%%%%%%%%%%%%%%%%%%%%%%%%%%%%%
\section{Objects and Types}
\begin{frame}[fragile]{Extendable object}
	\begin{block}{Extendable objects:  \lstinline!ExtObject!.}
		\begin{lstlisting}
var car = new ExtObject;
car.color := "red"; // := creates a new member
car.speed := 190;
car.outputDesc := fn() {
		out("This is a ", this.color, " car ");
		out("with top speed ", this.speed, ".\n");
};


car.speed = 185;
car.outputDesc();
		\end{lstlisting}
	\end{block}
	Output: This is a red car with top speed 185.
\end{frame}

\begin{frame}[fragile]{Inheritance}
	\begin{block}{Types and inheritance:}
		\begin{lstlisting}
var Shape = new Type;
Shape.color := "white";

// New type that is derived from Shape
var Polygon = new Type(Shape); 
Polygon.numVertices := 3;

// New type that is derived from Shape
var Circle = new Type(Shape); 
Circle.radius := 0;

var circle = new Circle;
circle.color = "red";
circle.radius = 5;
		\end{lstlisting}
	\end{block}
\end{frame}
 
\begin{frame}{Member attribute properties}
	\begin{block}{Example}
		\lstinputlisting{Properties.escript}
	\end{block}
\end{frame}

\section{Example}
%\begin{frame}{Delegation}
	%Call a function on another object.
	%\begin{block}{Example}
		%\lstinputlisting{Delegation.escript}
	%\end{block}
%\end{frame}


%\section{Examples}
%\begin{frame}{Factorial}
	%Factorial: $\qquad n! = 1 \cdot 2 \cdot 3 \cdot \ldots \cdot n \qquad 0! = 1$
	%\begin{block}{Example}
		%\lstinputlisting{ExampleFactorial.escript}
	%\end{block}
%\end{frame}

\begin{frame}{Player}
	\begin{block}{Example}
		\lstinputlisting{ExamplePlayer.escript}
	\end{block}
\end{frame}
%
%}
%%
%\begin{frame}{Outlook}
%
	%Part II: 
	%\begin{itemize}
	%\item	Std library continued...
		%\begin{itemize}
			%\item Dynamic object compositing using \textbf{traits}.
			%\item Organizing projects using \textbf{modules}.
			%\item Data handling using \textbf{DataWrapper}.
		%\end{itemize}
	%\item	Writing a C++ binding.
		%\begin{itemize}
			%\item Function wrapper.
			%\item Object wrapper.
		%\end{itemize}
	%\end{itemize}
%
%
%\end{frame}
%
%
%% binding parameters/Objects
%% var and static life
%% attributes
%% exception handling
%% std lib
%% modules
%% traits
%
%%homework
%%%%%%%%%%%%%%%%%%%%%%%%%%%%%%%%%%%%%%%%%%%%%%%%%%%%%%%%%%%%%%%%%%%%

\section{Loading and executing code}


\begin{frame}[fragile]{Compiler-constants}
\begin{itemize}
\item The EScript compiler defines compile time constants  (similar to C++ preprocessor defines).
\item The filename of the current EScript file: \lstinline!__FILE__!
\item The folder of the current EScript file: \lstinline!__DIR__!
\item The line number: \lstinline!__LINE__!
\item Main use cases: Debug output and loading files relative to the current file.
\end{itemize}
\end{frame}


\begin{frame}[fragile]{Loading script files}
\begin{itemize}
	\item EScript script files are loaded and executed using \lstinline!load( filename )!
	\item Files can return an object:
		\begin{lstlisting}	
	// File: main.escript
	out( load( __DIR__+"/"+test.escript ) );
	
	// File: test.escript
	outln( "Executing ",__FILE__);
	return "someResultValue";

	// output: someResultValue
		\end{lstlisting}	
\end{itemize}
\end{frame}

\begin{frame}[fragile]{Evaluating script code}
\begin{itemize}
\item EScript code can be executed using \lstinline!eval( "Code..." )!
\item The executed code can return an object:
		\begin{lstlisting}
outln(
	eval(	
		"outln('Executing code'); "
		"return 'someResult'; "
	)
);
		\end{lstlisting}
\end{itemize}
\end{frame}

\begin{frame}[fragile]{Variable injection}
\begin{itemize}
	\item To pass data to loaded or evaluated code, static variables can be injected.
	\item The variables are passed as additional Map-parameter to \lstinline!eval! or \lstinline!load!.
			\begin{lstlisting}

eval("outln('Value of myVar:',myVar); ",
	{	$myVar : 42 }
);

	// Output: Value of myVar: 42
		\end{lstlisting}
\end{itemize}
\end{frame}


%
%\begin{frame}{Todo}
%2.Teil Std-Lib
%- Load Std-Lib
%- Module-Konzept (Dependency injection)
%- Std-Namespace
%- MultiProcedure
%- DataWrapper
%- JSon-DataStore
%- Set/PriorityQueue
%- ObjectSerialization
%- addRevocably
%
%
%3.Teil PADrend
%- High-Level-Pluginstruktur
%- Was ist ein Plugin?
%- Was ist ein Extension-Point?
%- GUI (Popup, DataWrapper, Menu, Registry, Button)
%- Beispiel-Plugin
%\end{frame}

%
%%%%%%%%%%%%%%%%%%%%%%%%%%%%%%%%%%%%%%%%%%%%%%%
%
\begin{frame}{Further Documentation}
	You can find additional documentation in \texttt{EScript/docs/Introduction.html}. \\
	\hspace{2cm}\\
	Part II introduces the EScript Std-library...
\end{frame}

\end{document}