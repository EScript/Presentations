\documentclass[ucs,9pt]{beamer}

% Set input encoding to UTF-8.
\usepackage[utf8x]{inputenc}

% Set fonts.
\usepackage[T1]{fontenc}
\usepackage{textcomp}
\usepackage{tgtermes}
\usepackage{tgheros}
\usepackage{tgcursor}

% Set language.
\usepackage[ngerman]{babel}

\usepackage{microtype}
\usepackage{listings}

\title{EScript}
\subtitle{Kurzvorstellung einer Skriptsprache}
\author{Benjamin~Eikel}
\date{13. Februar 2012}

\useinnertheme{rectangles}
\useoutertheme{infolines}
\usecolortheme{seahorse}
\usecolortheme{rose}
\setbeamertemplate{bibliography item}[triangle] 

\begin{document}
\lstdefinelanguage{EScript}{
	morekeywords={var,new,fn,this,true,false,void,if,else,while,for,foreach,as,thisFn,break,continue,return},
	morekeywords=[2]{out,ExtObject,Type,Number},
	sensitive=true,
	morecomment=[l]{//},
	morecomment=[s]{/*}{*/},
	morestring=[b]{"},
	morestring=[d]{’}
}
\lstset{
	language=EScript,
	showstringspaces=false,
	tabsize=4,
	basicstyle=\ttfamily,
	keywordstyle=[2]\color{blue}
}
\maketitle

\begin{frame}{Übersicht}
\tableofcontents
\end{frame}

\section{Einführung}
\begin{frame}{Was ist EScript?}
EScript $\ldots$ \\[1em]
\begin{itemize}
	\addtolength{\itemsep}{\baselineskip}
	\item ist eine interpretierte, objektorientierte Skriptsprache.
	\item hat eine ähnliche Syntax wie C.
	\item wurde entwickelt, um C++-Objekte einfach in Skripten verwenden zu können.
	\item ist unter einer freien Softwarelizenz veröffentlicht.
	\item ist erhältlich unter \url{http://escript.berlios.de/}.
\end{itemize}
\end{frame}

\begin{frame}[fragile]{Erste Beispiele}
\begin{itemize}
	\addtolength{\itemsep}{\baselineskip}
	\item EScript-Dateien sollten die Endung \texttt{.escript} haben.
	\item Ein einfaches Skript:
			\begin{lstlisting}
				out("Hallo Welt!\n");
			\end{lstlisting}
\end{itemize}
\end{frame}

\section{Datentypen}
\begin{frame}[fragile]{Einfache Typen}
\begin{description}
	\addtolength{\itemsep}{\baselineskip}
	\item[Number] Beispiele: \lstinline!1!, \lstinline!27.4!, \lstinline!0x1a!, \lstinline!25 / 5!, \lstinline!3 + 4!
	\item[String] Beispiele: \lstinline!"ein"!, \lstinline!'beispiel'!, \lstinline!"hallo" + "welt"!
	\item[Bool] \lstinline!true! oder \lstinline!false!
	\item[Void] \lstinline!void!
\end{description}
\end{frame}

\begin{frame}[fragile]{Variablen, Kommentare}
	\begin{block}{var}
		\begin{lstlisting}
			/*
			   Declaring a variable is done using the keyword
			   "var", an identifier, an equation sign, and an
			   expression on the right side. The type of the
			   variable is deduced from the expression on the
			   right side.
			 */
			var xPos = 500 - 80 / 2;

			// The variable "message" will be of type String
			var message = "Please click the button";

			// Dynamically change the type to Number
			message = 5;
		\end{lstlisting}
	\end{block}
\end{frame}

\begin{frame}[fragile]{Funktionen}
	\begin{block}{fn}
		\begin{lstlisting}
			var square = fn(num) {
			    return num * num;
			};
			var a = square(5);
			var b = square(4.2);
		\end{lstlisting}
	\end{block}
\end{frame}

\begin{frame}[fragile]{Komplexere Typen (1)}
	\begin{block}{Array}
		\begin{lstlisting}
			var numbers = [3, 23, 7, 3, 100, 1, 35];
			var colors = ["red", "green", "blue"];
		\end{lstlisting}
	\end{block}
	\begin{block}{Map}
		\begin{lstlisting}
			var fruits = {
			    "lemon" : "yellow",
			    "cherry" : "red"
			};
			fruits["apple"] = "green";
		\end{lstlisting}
	\end{block}
\end{frame}

\begin{frame}[fragile]{Komplexere Typen (2)}
	\begin{block}{ExtObject}
		\begin{lstlisting}
			var car = new ExtObject();
			car.color := "red";
			car.speed := 190;
			car.outputDesc := fn() {
			    out("This is a ", this.color, " car ");
			    out("with top speed ", this.speed, ".\n");
			};

			...

			car.speed = 185;
			car.outputDesc();
		\end{lstlisting}
	\end{block}
	Output: This is a red car with top speed 185.
\end{frame}

\begin{frame}[fragile]{Komplexere Typen (3)}
	\begin{block}{Type}
		\begin{lstlisting}
			var Shape = new Type();
			Shape.color := "white";

			// Neuer Typ, der von Shape erbt
			var Polygon = new Type(Shape); 
			Polygon.numVertices := 3;

			// Neuer Typ, der von Shape erbt
			var Circle = new Type(Shape); 
			Circle.radius := 0;

			var circle = new Circle();
			circle.color = "red";
			circle.radius = 5;
		\end{lstlisting}
	\end{block}
\end{frame}
 
\section{Kontrollstrukturen}
\begin{frame}[fragile]{Abfragen}
	\begin{block}{if}
		\begin{lstlisting}
			var result = /* some function */;
			if(result) {
			    out("Success");
			} else {
			    out("Failure");
			}
		\end{lstlisting}
		\begin{lstlisting}
			var num = /* some number */;
			if(num < 0) {
			    out("Too small");
			} else if(num >= 0 && num <= 100) {
			    out("Range okay");
			} else {
			    out("Too large");
			}
		\end{lstlisting}
	\end{block}
\end{frame}

\begin{frame}[fragile]{Schleifen (1)}
	\begin{block}{while}
		\begin{lstlisting}
			var tasks = [/* some tasks */];
			while(!tasks.empty()) {
			    var firstTask = tasks.front();
			    tasks.popFront();
			    // do something with first task
			}
		\end{lstlisting}
	\end{block}
\end{frame}

\begin{frame}[fragile]{Schleifen (2)}
	\begin{block}{for}
		\begin{lstlisting}
			var sum = 0;
			for(var i = 0; i < 100; ++i) {
			    sum += i;
			}
			out("Sum of numbers: ", sum, "\n");
		\end{lstlisting}
	\end{block}
\end{frame}

\begin{frame}[fragile]{Schleifen (3)}
	\begin{block}{foreach}
		\begin{lstlisting}
			var chars = ["a", "c", "k", "b", "d", "x", "j"];
			foreach(chars as var c) {
			    if(c == "x") {
			        out("Character \"x\" found.");
			        break;
			    }
			}
		\end{lstlisting}
	\end{block}
\end{frame}

\section{Beispiele}

\begin{frame}[fragile]{Fakultät}
	Fakultät: $\qquad n! = 1 \cdot 2 \cdot 3 \cdot \ldots \cdot n \qquad 0! = 1$
	\begin{block}{Implementierung}
		\begin{lstlisting}
			var factorialRecursive = fn(Number n) {
			    if(n == 0) {
			        return 1;
			    }
			    return thisFn(n - 1) * n;
			};
			var factorialIterative = fn(Number n) {
			    var product = 1;
			    for(var i = 2; i <= n; ++i) {
			        product *= i;
			    }
			    return product;
			};
		\end{lstlisting}
	\end{block}
\end{frame}

\begin{frame}[fragile]{Spieler}
	\begin{block}{Implementierung}
		\begin{lstlisting}
			var Player = new Type();
			Player.x := 0;
			Player.y := 0;
			var movePlayer = fn(player, Number dx, Number dy) {
			    player.x += dx;
			    player.y += dy;
			};
			var printPos = fn(player) {
			    out("Player position: (", player.x);
			    out(", ", player.y, ")\n");
			};

			var playerA = new Player();
			movePlayer(playerA, 5, 7);
			printPos(playerA);
		\end{lstlisting}
	\end{block}
\end{frame}

\end{document}
