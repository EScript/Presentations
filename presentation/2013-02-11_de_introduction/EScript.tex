\documentclass[ucs,9pt]{beamer}

% Set input encoding to UTF-8.
\usepackage[utf8x]{inputenc}

% Set language.
\usepackage[ngerman]{babel}

\usepackage{microtype}
\usepackage{listings}

\title{EScript}
\subtitle{Kurzvorstellung einer Skriptsprache}
\author{Benjamin~Eikel}
\date{11. Februar 2013}

\newcommand{\HNIphone}{+49 5251 60-6452}
\newcommand{\HNIfax}{+49 5251 60-6482}
\newcommand{\HNIemail}{eikel@upb.de}
\newcommand{\HNIweb}{http://wwwhni.upb.de/en/alg/}
\newcommand{\HNIaffiliation}{Benjamin Eikel ist Stipendiat der \href{http://pace.uni-paderborn.de/de/studienprogramme/igs.html}{\includegraphics[width=.9\linewidth]{logo_igs}}}

% Activate the Heinz Nixdorf Institute theme.
\usepackage{HNI/beamerthemeHNI}

\AtBeginSection[] {
\begin{frame}{Übersicht}
	\tableofcontents[currentsection]
\end{frame}
}

\begin{document}
\lstdefinelanguage{EScript}{
	morekeywords={var,new,fn,this,true,false,void,if,else,while,for,foreach,as,thisFn,break,continue,return},
	morekeywords=[2]{out,outln,ExtObject,Type,Number},
	sensitive=true,
	morecomment=[l]{//},
	morecomment=[s]{/*}{*/},
	morestring=[b]{"},
	morestring=[d]{’}
}
\lstset{
	language=EScript,
	showstringspaces=false,
	tabsize=4,
	basicstyle=\ttfamily,
	keywordstyle=[2]\color{blue}
}
\maketitle

\begin{frame}{Übersicht}
\tableofcontents
\end{frame}

\section{Einführung}
\begin{frame}[t]{Was ist EScript?}
\includegraphics[width=1cm]{EScript_Logo} EScript $\ldots$ \\[1em]
\only<1>{
\begin{itemize}
	\addtolength{\itemsep}{\baselineskip}
	\item ist eine objektorientierte Skriptsprache.
	\item wird übersetzt und zur Laufzeit durch eine virtuelle Maschine ausgeführt.
	\item hat eine ähnliche Syntax wie C.
	\item wurde entwickelt, um C++-Objekte einfach in Skripten verwenden zu können.
\end{itemize}
}
\only<2>{
\begin{itemize}
	\addtolength{\itemsep}{\baselineskip}
	\item ist unter einer freien Softwarelizenz veröffentlicht.
	\item ist erhältlich unter \url{http://escript.berlios.de/}.
	\item kann mit CMake gebaut werden.
	\item hat einen Kommandozeileninterpretierer.
	\item kann intern von anderen C++-Projekten benutzt werden (z.\,B. PADrend).
\end{itemize}
}
\end{frame}

\begin{frame}[fragile]{Erstes Beispiel}
\begin{itemize}
	\addtolength{\itemsep}{\baselineskip}
	\item EScript-Dateien sollten die Endung \texttt{.escript} haben.
	\item Der EScript-Parser analysiert das Skript Zeile für Zeile.
	\item Ein einfaches Skript:
			\begin{lstlisting}
				out("Hallo Welt!\n");
			\end{lstlisting}
\end{itemize}
\end{frame}

\section{Datentypen}
\begin{frame}[fragile]{Einfache Typen}
	\begin{block}{Number}
		\hfill \lstinline!1! \hfill \lstinline!27.4! \hfill \lstinline!0x1a! \hfill \lstinline!25 / 5! \hfill \lstinline!3 + 4! \hfill{}
	\end{block}
	\begin{block}{String}
		\hfill \lstinline!"ein"! \hfill \lstinline!'beispiel'! \hfill \lstinline!"hallo" + "welt"! \hfill{}
	\end{block}
	\begin{block}{Bool}
		\hfill \lstinline!true! \hfill \lstinline!false! \hfill{}
	\end{block}
	\begin{block}{Void}
		\hfill \lstinline!void! \hfill{}
	\end{block}
\end{frame}

\begin{frame}[fragile]{Typkonvertierung}
	\begin{block}{Keine Konvertierung zu false}
		\begin{lstlisting}
			outln(false || false); // Ausgabe: false
			outln(false || 0); // Ausgabe: true
			outln(false || ""); // Ausgabe: true
		\end{lstlisting}
	\end{block}
	\pause
	\begin{block}{Konvertierung von String nach Number}
		\begin{lstlisting}
			outln((60 + "4").sqrt()); // Ausgabe: 8
			outln((10 * "10").log(10)); // Ausgabe: 2
		\end{lstlisting}
	\end{block}
	\pause
	\begin{block}{Konvertierung von Number nach String}
		\begin{lstlisting}
			outln("4" + 60); // Ausgabe: 460
			outln("12" + 3); // Ausgabe: 123
		\end{lstlisting}
	\end{block}
\end{frame}

\begin{frame}[fragile]{Variablen, Kommentare}
	\begin{block}{var}
		\begin{lstlisting}
			/*
			   Declaring a variable is done using the keyword
			   "var", an identifier, an equation sign, and an
			   expression on the right side. The type of the
			   variable is deduced from the expression on the
			   right side.
			 */
			var xPos = 500 - 80 / 2;

			// The variable "message" will be of type String
			var message = "Please click the button";

			// Dynamically change the type to Number
			message = 5;
		\end{lstlisting}
	\end{block}
\end{frame}

\begin{frame}[fragile]{Funktionen}
	\begin{block}{fn}
		\begin{lstlisting}
			var square = fn(num) {
			    return num * num;
			};
			var a = square(5);
			var b = square(4.2);
		\end{lstlisting}
	\end{block}
\end{frame}

\begin{frame}[fragile]{Komplexere Typen (1)}
	\begin{block}{Array}
		\begin{lstlisting}
			var numbers = [3, 23, 7, 3, 100, 1, 35];
			var colors = ["red", "green", "blue"];
		\end{lstlisting}
	\end{block}
	\begin{block}{Map}
		\begin{lstlisting}
			var fruits = {
			    "lemon" : "yellow",
			    "cherry" : "red"
			};
			fruits["apple"] = "green";
		\end{lstlisting}
	\end{block}
\end{frame}

\begin{frame}[fragile]{Komplexere Typen (2)}
	\begin{block}{ExtObject}
		\begin{lstlisting}
			var car = new ExtObject();
			car.color := "red";
			car.speed := 190;
			car.outputDesc := fn() {
			    out("This is a ", this.color, " car ");
			    out("with top speed ", this.speed, ".\n");
			};

			...

			car.speed = 185;
			car.outputDesc();
		\end{lstlisting}
	\end{block}
	Ausgabe: This is a red car with top speed 185.
\end{frame}

\begin{frame}[fragile]{Komplexere Typen (3)}
	\begin{block}{Type}
		\begin{lstlisting}
			var Shape = new Type();
			Shape.color := "white";

			// Neuer Typ, der von Shape erbt
			var Polygon = new Type(Shape); 
			Polygon.numVertices := 3;

			// Neuer Typ, der von Shape erbt
			var Circle = new Type(Shape); 
			Circle.radius := 0;

			var circle = new Circle();
			circle.color = "red";
			circle.radius = 5;
		\end{lstlisting}
	\end{block}
\end{frame}
 
\section{Kontrollstrukturen}
\begin{frame}[fragile]{Abfragen (1)}
	\begin{block}{if}
		\begin{lstlisting}
			var result = /* some function */;
			if(result) {
			    out("Success");
			} else {
			    out("Failure");
			}
		\end{lstlisting}
		\begin{lstlisting}
			var num = /* some number */;
			if(num < 0) {
			    out("Too small");
			} else if(num >= 0 && num <= 100) {
			    out("Range okay");
			} else {
			    out("Too large");
			}
		\end{lstlisting}
	\end{block}
\end{frame}

\begin{frame}[fragile]{Abfragen (2)}
	\begin{block}{? (conditional operator)}
		\begin{lstlisting}
			var num = /* some number */;
			var positive = (num > 0) ? true : false;
		\end{lstlisting}
	\end{block}
	\pause
	\vfill
	\emph{Hinweis:} Es gibt kein \lstinline!switch! in EScript.
\end{frame}

\begin{frame}[fragile]{Schleifen (1)}
	\begin{block}{while}
		\begin{lstlisting}
			var tasks = [/* some tasks */];
			while(!tasks.empty()) {
			    var firstTask = tasks.front();
			    tasks.popFront();
			    // do something with first task
			}
		\end{lstlisting}
	\end{block}
\end{frame}

\begin{frame}[fragile]{Schleifen (2)}
	\begin{block}{for}
		\begin{lstlisting}
			var sum = 0;
			for(var i = 0; i < 100; ++i) {
			    sum += i;
			}
			out("Sum of numbers: ", sum, "\n");
		\end{lstlisting}
	\end{block}
\end{frame}

\begin{frame}[fragile]{Schleifen (3)}
	\begin{block}{foreach}
		\begin{lstlisting}
			var chars = ["a", "c", "k", "b", "d", "x", "j"];
			foreach(chars as var i, var c) {
			    if(c == "x") {
			        out("Character \"x\" found at index " + i);
			        break;
			    }
			}
		\end{lstlisting}
	\end{block}
	Ausgabe: Character "x"{} found at index 5
\end{frame}

\section{Weitere Funktionalität}
\begin{frame}{Delegation}
	Aufruf einer Funktion auf einem anderen Objekt.
	\begin{block}{Beispiel}
		\lstinputlisting{Delegation.escript}
	\end{block}
\end{frame}

\begin{frame}{Attribute}
	\begin{block}{Beispiel}
		\lstinputlisting{Properties.escript}
	\end{block}
\end{frame}

\section{Beispiele}
\begin{frame}{Fakultät}
	Fakultät: $\qquad n! = 1 \cdot 2 \cdot 3 \cdot \ldots \cdot n \qquad 0! = 1$
	\begin{block}{Beispiel}
		\lstinputlisting{ExampleFactorial.escript}
	\end{block}
\end{frame}

\begin{frame}{Spieler}
	\begin{block}{Beispiel}
		\lstinputlisting{ExamplePlayer.escript}
	\end{block}
\end{frame}

\begin{frame}{Zusätzliche Dokumentation}
	Zusätzliche Dokumentation befindet sich in \texttt{EScript/docs/Introduction.html}.
\end{frame}

\HNIlastframe

\end{document}